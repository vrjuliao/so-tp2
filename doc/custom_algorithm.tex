O algoritmo de substituição proposto por esse trabalho, foi baseado na criação
de um \textit{hashing} de endereços.
Dentre os argumentos recebidos pelo programa, temos a quantidade de memória
disponível ($q$) e o tamanho de cada página ($k$).
Então, sendo $l = q\div k$ a quantidade de páginas gerenciadas pelos métodos
de substituição, aplicamos uma função que mapeia qualquer endereço para um
número entre $0$ e $l-1$.
Assim, o método de substituição criado por esse trabalho, trabalhará com a
quantidade correta de páginas.
Esse hashing acontece de modo a aplicar uma operação booleana \textit{and}
bit a bit, entre o endereço da página requisitada e $l-1$.
Como $l$ será sempre um valor em potência de 2, temos que $l-1$, quando
transformado para a base 2, é um número preenchido apenas com 1.

Dessa maneira o hashing obtém os bits menos significativos do endereço de uma
página e assim, pode-se consultar nesse array de $l$ endereços, se determinada
página requisitada já está na memória ou não.
É importante informar que essa operação booleana é aplicada sobre o endereço
da página, ou seja, o endereço dado no arquivo de entrada passa por uma
transformação de shifting e assim é obtido o endereço da página, e só depois
disso aplica-se a operação \textit{and} bit a bit com $l-1$.
O hashing se dá da seguinte maneira:

\begin{algorithm}
    \caption{Hashing Transformation}\label{alg:transformation}
    \begin{algorithmic}[1]
    \Function{Hashing Transformation}{$Page\ Address\ PAddr, Pages\ Quantity\ l$}
        
    \Return $PAddr\ \&\&\ (l-1)$   
    \EndFunction
\end{algorithmic}
\end{algorithm}

Para melhor esclarecer, observe o seguinte exemplo:
Dada um programa com uma quantidade de memória disponível igual a 128 Kilo
Bytes, em que cada página possui 8 Kilo Bytes de memória disponível.
Nesse caso, teremos então $128 \div 8 = 16$ elementos na tabela de páginas,
logo $l = 16$.
Então suponha que o programa deseje acessar uma página $x$ em que o valor
decimal do endereço será $550$.
Portanto, dado um array $A$ que armazena os endereços das páginas disponíveis
na memória, em que o tamanho de $A$ é igual a $l$, poderemos mapear qualquer
endereço de página para uma posição em $A$ através da operação booleana
previamente citada.
Observe o mapeamento da página $x$:
$$
\langle\textit{and}\rangle\frac{x\ \ \rightarrow 550_{10}\ = 1000100110_2}{l-1 \rightarrow 15_{10} = 0000001111_2}
= 0000000110_2 = 12_{10}
$$

Por fim, a página de que tem o endereço igual a 550 será armazenada no índice
12 do vetor $A$, sendo que este vetor é indexado com o primeiro elemento na
posição 0.
Para executar o algoritmo criado por este trabalho, o argumento de execução
que contem o nome do método de reposição a ser executado dever ser 
\texttt{custom}.