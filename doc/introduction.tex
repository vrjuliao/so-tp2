Sistemas operacionais, precisam prover acesso à memória para os diversos
programas em execução.
Porém, memória é um recurso limitado e que deve ser gerenciado de maneira a
atender todos os processos em execução.
Essa tarefa necessita de um mecanismo capaz de realizar tal controle,
permitindo que programas acessem à memória e também ao disco, entretanto
abstraindo do processo em execução a necessidade de identificar se determinado
endereço de memória está no disco ou na memória volátil da máquina.
Tal função é conhecida como virtualização de memória e nisso encontra-se a
necessidade de um algoritmo que crie uma tabela na qual liste as páginas que um
processo possui armazenadas em memória, ou em disco.
Essa tabela possui um espaço limitado, então se um endereço (virtual)
encontra-se em disco, cabe ao mecanismo de virtualização a busca desses dados e
a colocação na memória física.

Quando determinado programa já ocupou todo o espaço reservado para a própria
execução, é necessário iniciar o salvamento dos dados em disco.
Essa ação leva à necessidade de realizar trocas entre o disco e a memória
volátil; sendo que o responsável pela troca é o algoritmo de reposição.
Por fim, para este trabalho, foi proposto o desenvolvimento de quatro
algoritmos de reposição: \textit{LRU}, \textit{Second Chance}, \textit{FIFO}
e um quarto algoritmo \textit{Custom}, que seria proposto pelo próprio aluno.
Dado o ambiente virtual do trabalho, tomaremos métricas diferentes do tempo de
execução para avaliar o desempenho de cada um dos métodos.
O ambiente simulado não executa leitura e escrita em disco assim como acontece
no ambiente real, portanto, aqui o tempo não é um grande fator de impacto.
Logo, as métricas avaliadas consistem na quantidade de páginas não encontradas
na memória volátil (\textit{page-faults}) e a quantidade de páginas sujas 
(\textit{dirty}) -- tais métricas identificam a quantidade de leituras e
escritas feitas no disco, respectivamente.

Os códigos apresentados por esse trabalho foram desenvolvidos na linguagem C e
testados no sistema operacional Ubuntu 16.04 e 18.04.
Uma vez que a linguagem permite a criação de tipos abstratos de dados, tal
conceito fora aplicado para instanciar as entidades necessárias no controle dos
métodos de substituição.